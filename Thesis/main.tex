%%%%%%%%%%%%%%%%%%%%%%%%%%%%%%%%%%%%%%%
% プリアンブル 各種設定情報
%%%%%%%%%%%%%%%%%%%%%%%%%%%%%%%%%%%%%%%
\documentclass[a4paper,11pt,oneside,openany]{jsbook}
\usepackage{docmute}
%%%%%%%%%スタイルファイルを追加したい場合にはmyjlabthesisstyleファイル内に記述すること%%%%%%%%%%%%%
\usepackage{myjlabthesisstyle}
\makeindex

%%%%%%%%%%基本情報設定変更の必要なし%%%%%%%%%%%%%%%%%%%%%%%%%%%%
\daigaku{青山学院大学}
\gakubu{社会情報}
\gakka{社会情報学科}
\syubetsu{卒業論文}
\labname{宮治研究室}
\chiefexaminer{宮治~~裕~~教授}

%%%%%%%%%%%%%%%%%%%%%%%%%%%%%%%%%%%%%%%
% ここから先「ここまで個人設定」の範囲に
% 各自の固有の情報を記入して下さい
%%%%%%%%%%%%%%%%%%%%%%%%%%%%%%%%%%%%%%%
\nendo{2023年度}
\teisyutsu{2024年~~1月}
\snum{15387019}
\jname{宮治~~裕}
\thesistitle{宮治研における論文作成について} %タイトルを記入
%\thesissubtitle{\LaTeX の利用} %サブタイトルを記入 ない場合はコメントアウト
%\SUBTtrue %サブタイトル有りの場合 ない場合は,コメントアウト
\SUBTfalse %サブタイトル無しの場合 有る場合は,コメントアウト
%%%%%%%%%% ここまで個人設定 %%%%%%%%%%%%%%


%%%%%%%%%%%%%%%%%%%%%%%%%%%%%%%%%%%%%%%
% ここから先,論文内原稿
% 「ここまで共通」まで編集不要
%%%%%%%%%%%%%%%%%%%%%%%%%%%%%%%%%%%%%%%
\begin{document}
\linesparpage{30} %行数指定
\mojiparline{35} %文字数指定
\pagestyle{empty}
\maketitle

\frontmatter
%%% 論文要旨
\chapter*{論文要旨}
%\thispagestyle{empty}
\addcontentsline{toc}{chapter}{論文要旨}
% 要旨には,論文の要約を記述します.要約と言っても全ての章や項目を均等に縮めるのではなく,必要な項目に絞って端的に示します.

% 論文の概要が,要旨に書かれた文章のみで伝わるようにしなければなりません.
% したがって,少なくとも「ざっくりとした背景」「研究の目的」「他の研究との違い/関わり」「構築したシステム/提案した事項」「結果/得られた結論」が書かれている必要があります.

日本は災害が多く発生する国であり,訪日観光客に対する災害対策の研究には需要がある.
本研究は訪日観光客向けの災害対策の1つとして,災害の情報を提供する情報システムを提案する.
研究の目的は予測できる災害に対して,訪日観光客が災害の影響を受ける前に防災行動を促すことを目的とする.
予測できる災害とは気象に基づく災害であり,本研究では風水害としている.\par
現状,訪日観光客は災害の情報の多くを日本のテレビやラジオから受け取っている.
これは日本住民に向けた情報であり,一部の訪日観光客は災害の情報を正しく理解できていない.
この問題点は,情報を受け取る人がある程度日本の災害に対して知識を持っている前提で提供されていることである.
そのため,事前に予測できる災害が起こることを認知していたのにも関わらず,災害で被害を受けることを想起できない.
よって,被害にあってから防災行動を取ることを余儀なくされてしまう現象が確認されている.\par
本研究では訪日観光客が予想できる災害の現象を想像するのに必要とする知識を定義する.
さらに,訪日観光客の旅程をもとに災害の動向を監視することで,災害の危険性をある程度認められる場合に適切な災害の情報を提供する.
本研究の新規性は訪日観光客が必要とする情報を定義した点,旅程に基づいて災害情報を提供する点である.
提案するシステムは訪日観光客が旅程を入力するアプリである.
気象庁の早期注意情報をもとに,旅程が災害の被害にあう可能性が予測されたとき,災害の情報を訪日観光客に通知する.\par
本システムが提供する情報と気象庁の天気予報の情報を比較する実験をおこなった.
評価指標は防護動機理論に基づいて決定され,脅威評価とした.
比較実験の結果,システムから得られた脅威評価の方が天気予報から得られた評価よりも高いことが判明した.
\pagestyle{plain}
\pagenumbering{roman}
% abstract.texの中は \chapterなど書かずに単なるテキストを入力する

%%% 謝辞
\chapter*{謝辞}
%\thispagestyle{empty}
\addcontentsline{toc}{chapter}{謝辞}
本研究を進めるにあたり,多くの方々にご指導ご調達を賜りました.
まず,指導教員の宮治裕教授からは多大なご指導を賜りました.
論文のテーマのアイデアや論理の組み立て方,困ったときに手助けをいただきました.感謝の念に堪えません.ありがとうございました.
また,実験の実施にあたり,宮治研究室の皆様と,同期の方々,青山学院大学の後輩たちには実験参加者を務めてください,貴重なデータ収集にご協力いただいたことを感謝いたします.
さらに,本研究で提案するシステムを開発するにあたり,多くのオープンソースのソフトウェアを利用しました.ソフトウェアの開発者たちに感謝いたします.
最後に,本研究ならびに学業全般にわたって経済的・心身的に支援してくださる家族に深く感謝し,お礼を申し上げます.
% thanks.texの中は \chapterなど書かずに単なるテキストを入力する

%%% 目次
\tableofcontents
% 目次は自動生成される
%
\mainmatter
\pagestyle{fancy}
\pagenumbering{arabic}
%%%%%%%%%% 「ここまで共通」 %%%%%%%%%%%%%%


%%%%%%%%%%%%%%%%%%%%%%%%%%%%%%%%%%%%%%%
% ここから先「ここまで論文本文」の範囲を
% 各自の章立てや付録にあわせて編集して下さい
%%%%%%%%%%%%%%%%%%%%%%%%%%%%%%%%%%%%%%%

%%% 本文ここから 同様に必要なだけ章を入れる
\documentclass[a4paper,11pt,oneside,openany]{jsbook}
\usepackage{myjlabthesisstyle}
\daigaku{青山学院大学}
\gakubu{社会情報}
\gakka{社会情報学科}
\syubetsu{卒業論文}
\labname{宮治研究室}
\chiefexaminer{宮治~~裕~~教授}

%%%%%%%%%%%%%%%%%%%%%%%%%%%%%%%%%%%%%%%
% ここから先「ここまで個人設定」の範囲に
% 各自の固有の情報を記入して下さい
%%%%%%%%%%%%%%%%%%%%%%%%%%%%%%%%%%%%%%%
\nendo{2023年度}
\teisyutsu{2024年~~1月}
\snum{15387019}
\jname{宮治~~裕}
\thesistitle{宮治研における論文作成について} %タイトルを記入
%\thesissubtitle{\LaTeX の利用} %サブタイトルを記入 ない場合はコメントアウト
%\SUBTtrue %サブタイトル有りの場合 ない場合は,コメントアウト
\SUBTfalse %サブタイトル無しの場合 有る場合は,コメントアウト
%%%%%%%%%% ここまで個人設定 %%%%%%%%%%%%%%

\begin{document}

\chapter{はじめに}
本論文では,○○○を△△△することにより,□□を明らかとする研究について記述する.

まず,本研究をおこなう背景となった事柄について述べる.
次に,研究目的の詳細を記述した後,類似研究との相違や関連研究とのつながりについて解説する.
また,次章以降の本論文の構成についてその概略を述べる.

\section{背景}
「背景」には,研究の目的の妥当性を示す事項を説明する.
個人的な興味や関心を書くのではなく,客観的な視点で記述する.
つまり,その説明には参考文献やデータを参照することが必要となる.

なお,背景をあまり詳しく書きすぎると,2章や3章などで書く内容が無くなったり重複するおそれがある.
研究の目的の妥当性につながる程度の内容(詳細さ)でかまわない.

\section{研究目的}
背景によって,研究の大きな目的が導かれる.
その大きな目的を正確に定義した後,本研究にて実際にターゲットとする目的を詳細に記述する\footnote{大きな目的は1年間の研究ではカバーしきれないため}.

また,背景にて実際の詳細なターゲットの必要性を示した場合には,それの詳細な条件を記載する.

\section{関連研究}
類似研究(同じような研究)とは,どこが違うのか(ターゲット,手法,想定結果など)を述べる必要がある.
また,参考にする先行研究(他組織の研究でも良い)とどのような関連性があるのかを述べる.

場合によっては,関連研究が研究目的より先に書いてあった方が「ながれ」が良い場合もある.
また,関連研究を背景の中に入れてしまった方が良いケースもある.
これらについては,文章を書きながら,判断するしかない.

\section{論文構成}
論文構成では,2章以降の大まかな記述内容の流れを示す.

たとえば,以下の様に記述する.
2章では,本研究にて活用した技術や関連サービスについて解説する.
3章では,提案・構築したシステムについて詳説する.
4章では,システムの有用性を検証する為に行った実験について記述する.
最後に5章において,本研究についてまとめ,今後の課題について述べる.


%%%%%%%%%%%%フォーマットの確認開始%%%%%%%%%%%%
\newpage
%\clearpage
\noindent
一二三四五六七八九零一二三四五六七八九零一二三四五六七八九零一二三四五\\
二\\
三\\
四\\
五\\
六\\
七\\
八\\
九\\
零\\
一\\
二\\
三\\
四\\
五\\
六\\
七\\
八\\
九\\
零  行数と列数の設定テスト 30行×35文字 = 1050文字/ページ\\
一\\
二\\
三\\
四\\
五\\
六\\
七\\
八\\
九\\
零
%%%%%%%%%%%%フォーマットの確認終了%%%%%%%%%%%%

%
\end{document}
 % 1章
\documentclass[a4paper,11pt,oneside,openany]{jsbook}
\usepackage{myjlabthesisstyle}
\daigaku{青山学院大学}
\gakubu{社会情報}
\gakka{社会情報学科}
\syubetsu{卒業論文}
\labname{宮治研究室}
\chiefexaminer{宮治~~裕~~教授}

%%%%%%%%%%%%%%%%%%%%%%%%%%%%%%%%%%%%%%%
% ここから先「ここまで個人設定」の範囲に
% 各自の固有の情報を記入して下さい
%%%%%%%%%%%%%%%%%%%%%%%%%%%%%%%%%%%%%%%
\nendo{2023年度}
\teisyutsu{2024年~~1月}
\snum{15387019}
\jname{宮治~~裕}
\thesistitle{宮治研における論文作成について} %タイトルを記入
%\thesissubtitle{\LaTeX の利用} %サブタイトルを記入 ない場合はコメントアウト
%\SUBTtrue %サブタイトル有りの場合 ない場合は,コメントアウト
\SUBTfalse %サブタイトル無しの場合 有る場合は,コメントアウト
%%%%%%%%%% ここまで個人設定 %%%%%%%%%%%%%%

\begin{document}

\chapter{宮治研用 \LaTeX スタイルパッケージの使い方}
Microsoft Word やその他のワープロソフトを利用して論文を書いても構わない.
しかしながら宮治研究室では,最終的には \LaTeX によってフォーマットを整えし,PDF化された論文を提出する.

本章では, \LaTeX で論文を書く際の各種設定などを宮治研究室用に調整したスタイルファイルの利用方法について記述する.

\section{システム概要}
システムの概要に関して述べる.
システムの全体概要は図 \ref{fig:system_summary}のようになっている.
本システムはクライアント部,バッチ処理部,データベース部の3部構成である.
各構成ごとに役割と機能の概要を説明する.
最後に処理の流れを述べる.

\begin{figure}[h]
  \centering
  \includegraphics[height=6cm]{figure31.pdf}
  %\vspace{-3mm}
  \caption{システムの全体概要図}
  \label{fig:system_summary}
  %\vspace{2mm}
\end{figure}

\subsection{クライアント部}
クライアント部はこのシステムにおけるユーザインタフェースの役割を持つ.
旅程データの入力機能と注意情報を提供する機能がある.
具体的にはスマートフォンのアプリケーションを開発した.

\subsubsection{旅程データの入力機能}
ユーザはまず旅程のタイトルと期間を入力し,旅程パッケージを作成する.
旅程パッケージとは旅程データを1つにまとめたものである.
旅程データには,場所データと交通機関のデータの2種類が存在する.ユーザはこの2種類のデータで旅程を構築する.
ユーザは訪れる予定の場所や利用する予定の交通機関の位置情報を,予定時刻とともに入力インタフェースを通して入力する.
位置情報は両方の種類においても都道府県と市区町村名の名称である.
インタフェースは旅程パッケージごとに入力されたデータをデータベースへ保存する.
保存する際には旅程パッケージにクライアントidを付与する.

\subsubsection{注意情報提供機能}
第1章で定義したストック情報とフロー情報をユーザに提供する.
旅程データの入力機能にて事前に入力した旅程データのうち,災害の影響を受けると判断された旅程データを表示する.
その際に影響を受ける災害の種類と可能性の確率を提供する.
さらに,ユーザに対して災害の種類ごとにその現象と対処法についての情報を提供する.
また,影響を受ける旅程データが交通機関である場合,災害の可能性を交通機関が運休する可能性として情報を提供する.
その際に,運休によって起こる現象についての情報を提供する.

\subsection{データベース部}
データベース部はこのシステムにおけるデータを管理する役割を持つ.
データを保存する機能とデータを提供する機能がある.
具体的にはクラウドデータベースを利用した.

\subsubsection{データ保存機能}
ユーザがクライアント部で入力した旅程データを旅程パッケージごとに保管する.
また,バッチ処理部で旅程データと紐付けられた注意情報も保管する.

\subsubsection{データ提供機能}
クライアント部の要求には,旅程パッケージに付与されているクライアントidを条件にデータを提供する.
バッチ処理部の要求には,旅程データの予定時刻を条件にデータを提供する.

\subsection{バッチ処理部}
バッチ処理部は,旅程データが災害の影響を受けるか監視する役割を持つ.
気象庁が定期的に公表する,注意情報を取得して,ユーザがクライアント部で入力した旅程データと照合する.
災害の影響を受ける旅程データがあれば注意情報と旅程データを紐づけてデータベースに保存する.
旅程データを作成したクラアントに通知する指示を出す.
具体的にはPythonのプログラムを作成した.

\subsubsection{気象庁から注意情報の取得}
注意情報とは災害の影響を受ける可能性と予定時刻,発表地域で構成される情報である.
この情報は気象庁から定期的にweb上で発表されている.
その時刻に合わせてバッチ処理が実行され,注意情報を取得する.

\subsubsection{旅程データの取得}
注意情報の予定時刻と重なる旅程データを検索する.
データベース部から旅程データを取得する.

\subsubsection{注意情報と旅程データを照合}
注意情報の予報時刻に当てはまる旅程データを抽出した後,旅程データの場所情報をもとに,注意情報を関連付けて保存するか判断する.
注意情報の発表地区は気象庁が独自に定めている,各都道府県の市区町村のまとまりである.
注意情報に災害の予測が記載されている場合,旅程データの位置情報から発表地区に属するデータを抽出し,注意情報を旅程データと紐付ける.
災害の影響を受けると判断された旅程データはデータベースに保存する.
もし注意情報に災害の予測が記載されてなければ何もしない.


\subsubsection{クライアント部へ通知指示}
災害の影響を受けると判断された旅程データを作成したクライアントに通知を送信するよう指示を出す.
しかし,評価実験をするにあたって評価指標とは関連がない機能であったことと,コスト削減のため本研究ではクライアント部からの通知機能を実装した.

\subsection{処理の流れ}
システムの処理の流れについて述べる.
図 \ref{fig:activity}にアクティビティを示す.

\begin{quote}
  \begin{enumerate}
    \item 旅程データの入力(クライアント部)
    \item 旅程データの保存(データベース部)
    \item 気象庁から注意情報の取得(バッチ処理部)
    \item 旅程データの取得(バッチ処理部)
    \item 旅程データの提供(データベース部)
    \item 注意情報と旅程データを照合(バッチ処理部)
    \item 注意情報の保存(データベース部)
    \item クライアント部へ通知指示(バッチ処理部)
    \item 通知(クライアント部)
    \item 注意情報の提供(データベース部)
    \item 注意情報提供(クライアント部)
  \end{enumerate}
\end{quote}

\begin{figure}[H]
  \centering
  \includegraphics[height=10cm]{figure32.jpg}
  %\vspace{-3mm}
  \caption{システムのアクティビティ図}
  \label{fig:activity}
  %\vspace{2mm}
\end{figure}
\section{環境}

本研究のシステム環境を構成ごとに述べる.

\begin{quote}
  \begin{itemize}
    \item クライアント部
    \begin{itemize}
      \item デバイス
      \begin{itemize}
        \item Mac Book Air (13-inch, 2020)
        \item iPhone 15
      \end{itemize}
      \item OS
      \begin{itemize}
        \item macOS Sonoma 14.0
        \item iOS 17.12
      \end{itemize}
      \item 開発言語・ソフトウェア
      \begin{itemize}
        \item Dart 3.0.5
        \item Flutter 3.10.5
        \item Xcode 15.1
      \end{itemize}
    \end{itemize}
    \item バッチ処理部
    \begin{itemize}
      \item OS
      \begin{itemize}
        \item Ubuntu 20.04
      \end{itemize}
      \item 開発言語・ソフトウェア
      \begin{itemize}
        \item Python 3.10.4
        \item Docker 20.10.21
      \end{itemize}
    \end{itemize}
  \end{itemize}
\end{quote}


\section{利用技術}

本研究のシステムに利用した技術を構成ごとに述べる.

\subsection{クライアント部}
クライアント部で利用した技術に関して述べる.

\subsubsection{Flutter}
Flutter\cite{Flutter}とは,2017年にGoogleによって作られたマルチプラットフォームアプリケーションの開発フレームワークである.
本研究では,iOSとAndroid両方のプラットフォームで動作するモバイルアプリケーションを開発するために利用した.

\subsubsection{Google Place API}
Google Place API\cite{GoogleMap}はGoogle Mapに登録されている位置情報の検索,詳細を取得する機能を提供している.
本研究では,旅程の場所データを入力する機能にGoogle Place APIを利用していている.
APIのPlace Autocompleteとという機能を利用して,場所の名前の検索語から候補を検索する機能を開発した.
ユーザは候補の中から場所を選択する.モバイルアプリは選択した場所のidを条件にしてPlace Detailsという機能で場所の詳細情報を取得する.
詳細情報は場所の名称と都道府県名,市区町村名,住所,場所アイコンのurlである.

\subsubsection{鉄道データ}
鉄道データ\cite{TrainData}は国土交通省が提供している鉄道に関するデータである.
内容は全国の旅客鉄道・軌道の路線や駅について,形状,鉄道区分,事業者,路線名,運営会社等を整備したものである.
本研究では交通機関の旅程データを入力する機能にこのデータを利用している.
事前にこのデータから路線名とその路線に属する駅の名称と緯度経度情報を抽出している.
その緯度経度情報をジオコーディングして,各駅ごとに属する都道府県と市区町村名のデータを作成する.
入力機能では,まず路線名から駅名の一覧を抽出して表示する.
そして,駅ごとの都道府県と市区町村名の情報を保存できるようにデータを加工してアプリに内蔵している.

\subsubsection{位置参照情報}
位置参照情報\cite{Geocoding}は国土交通省が提供しており,全国の都市計画区域相当範囲を対象に,街区単位の位置情報を整備したデータである.
位置情報とは代表点の緯度,経度,平面直角座標である.
ジオコーディングは入力した緯度経度に一番近しい代表点の都道府県名と市区町村名とする.
このデータを利用して,鉄道データから入手した駅の緯度経度情報からジオコーディングをするのに利用した.

\subsubsection{外部ライブラリ}
Flutterで利用した外部ライブラリを以下に示す.

\begin{quote}
  \begin{itemize}
    \item google\_maps\_webservice: 0.0.20-nullsafety.5
    \item cloud\_firestore: 4.8.4
    \item flutter\_local\_notifications: 15.1.0+1
  \end{itemize}
\end{quote}

\subsection{データベース部}
データベース部で利用した技術に関して述べる.

\subsubsection{Cloud Firestore}

Cloud Firestore\cite{Firebase}は,モバイルアプリケーションやウェブアプリのデータ保存,同期,照会ができるNoSQLデータベースである.
本研究では,旅程データを保存するデータベースとして利用した.

\subsection{バッチ処理部}
バッチ処理部で利用した技術に関して述べる.

\subsubsection{気象庁防災情報XML}

気象庁防災情報XML\cite{KishoutyouXML}は,気象庁が発表する防災情報をXML形式で表現したデータである.\par
早期注意情報(警報級の可能性)という防災情報のXMLを利用した.
この情報は雨や風などの災害種類ごとに,警報級の災害の被害に遭う確率を「高」「中」の2段階で提供している.
警報級の災害とは,気象庁が発表する防災応報の警報基準以上の災害を意味する.
警報は重大な災害が起こる恐れのある旨を警告して行う予報である.
大雨に関する警報は警戒レベル3と対応付けられており,日本住民の避難行動の目安となっている.\par
早期注意情報(警報級の可能性)のXMLには,翌日までの情報と2日先から5日先までの情報の2種類のXMLが存在する.
翌日までの情報は毎日5時,11時,17時に定期的に発表されていて,2日先から5日先までの情報は毎日11時,17時に定期的に発表されている.
このXMLデータから災害の影響が予測される地区,災害の可能性,災害の種類の情報を抽出して注意情報とした.
予報地区は2種類のXMLごとに違う区分で分割されているが,どちらも都道府県かいくつかの市区町村のまとまりを単位としている.\par
本研究では,これらの情報を定期取得して,フロー情報として利用している.

\subsubsection{cron}
cronはUNIX系のOSで標準的に利用されているDeamonの一種で,設定したスケジュールに沿って指定されたプログラムを定期実行するものである.
本研究では,バッチ処理の定期実行のために利用した.

\subsubsection{外部ライブラリ}
Pythonで利用した外部ライブラリを以下に示す.

\begin{quote}
  \begin{itemize}
    \item firebase-admin: 6.2.0
    \item requests: 2.31.0
  \end{itemize}
\end{quote}
\section{使用方法}

本研究のシステムの使用方法を述べる.使用手順は以下の通りである.

\begin{quote}
  \begin{enumerate}
    \item アプリを起動する
    \item 旅程パッケージを作成する
    \item 旅程データを作成する
    \item 通知を受け取る
    \item 災害注意予報を確認する
    \item ストック情報を確認する
  \end{enumerate}
\end{quote}

\subsection{アプリを起動する}
アプリを利用するのに,旅行計画の作成が必要であるため,機能メニューの旅行計画をタップする.
ホーム画面はアプリを立ち上げたとき,最初に表示される画面である.
この画面ではアプリの概要について説明が閲覧できる.
この画面から予報についての画面と旅程パッケージの作成画面に遷移できる.

\begin{figure}[H]
  \centering
  \includegraphics[height=10cm]{./fig/normal_home_screen.png}
  %\vspace{-3mm}
  \caption{通常時のホーム画面}
  \label{fig:normal_home_screen}
  %\vspace{2mm}
\end{figure}

% \subsection{予報についての画面}
% 予報についての画面はアプリケーションが提供する災害注意予報について説明している画面である。
% アプリでの予報の表示の仕方や、気象庁防災情報XMLについて簡単に説明している。

% \begin{figure}[H]
%   \centering
%   \includegraphics[height=8cm]{./fig/forcast_screen_1.png}
%   %\vspace{-3mm}
%   \caption{予報についての画面の図1}
%   \label{fig:forecast_screen_1}
%   %\vspace{2mm}
% \end{figure}

% \begin{figure}[H]
%   \centering
%   \includegraphics[height=8cm]{./fig/forcast_screen_2.png}
%   %\vspace{-3mm}
%   \caption{予報についての画面の図2}
%   \label{fig:forecast_screen_2}
%   %\vspace{2mm}
% \end{figure}

\subsection {旅程パッケージを作成する}
旅程データを作成するために旅程パッケージを作成する.
旅程パッケージ画面は作成した旅程パッケージの一覧と旅行パッケージの作成機能を提供している画面である.
旅行パッケージは右下の計画の作成ボタンをタップすると旅行パッケージ作成画面に遷移して作成できる.
旅行パッケージの一覧からパッケージをタップすると,旅程データ画面に遷移する.

\begin{figure}[H]
  \begin{minipage}[b]{0.45\linewidth}
    \centering
    \includegraphics[height=10cm]{./fig/travel_pack_list.png}
    %\vspace{-3mm}
    \caption{旅程パッケージ画面}
    \label{fig:travel_pack_list}
    %\vspace{2mm}
  \end{minipage}
  \begin{minipage}[b]{0.45\linewidth}
    \centering
    \includegraphics[height=10cm]{./fig/travel_pack_create.png}
    %\vspace{-3mm}
    \caption{旅程パッケージ作成画面}
    \label{fig:travel_pack_create}
    %\vspace{2mm}
  \end{minipage}
\end{figure}

\subsection {旅程データを作成する}
旅程パッケージに旅程データを作成し,登録する.
旅程データ画面は日付ごとに作成した旅程データの一覧と場所・交通データの作成機能を提供している画面である.
下部のAdd Spotボタンをタップすると場所データの作成が,Add Transportationボタンをタップすると交通データの作成画面に遷移する.

\begin{figure}[H]
  \centering
  \includegraphics[height=10cm]{./fig/travel_data_list.png}
  %\vspace{-3mm}
  \caption{旅程データ画面}
  \label{fig:travel_data_list}
  %\vspace{2mm}
\end{figure}

\subsubsection {場所データの作成}
場所データを作成する.
訪れる予定の場所を検索する機能とその時刻を入力する機能を提供している画面である.
検索機能はGoogle Map Apiを利用している.
検索すると場所の候補が複数表示されるので,1つを選択してSaveボタンを押す.

\begin{figure}[H]
  \centering
  \includegraphics[height=10cm]{./fig/spot_data_save.png}
  %\vspace{-3mm}
  \caption{場所データの作成画面}
  \label{fig:spot_data_save}
  %\vspace{2mm}
\end{figure}

\subsubsection {交通機関データの作成}
交通データを作成する.
利用する予定の鉄道の路線を検索する機能を提供する.
路線を選択した後,路線に属する駅の一覧が表示される.
自分が利用する予定の駅を1つ以上選択し,時刻とともにデータを保存する.

\begin{figure}[H]
  \begin{minipage}[b]{0.45\linewidth}
    \centering
    \includegraphics[height=10cm]{./fig/railway_search.png}
    %\vspace{-3mm}
    \caption{鉄道の路線を検索する画面}
    \label{fig:railway_search}
    %\vspace{2mm}
  \end{minipage}
  \begin{minipage}[b]{0.45\linewidth}
    \centering
    \includegraphics[height=10cm]{./fig/trans_data_save.png}
    %\vspace{-3mm}
    \caption{利用する駅を保存する画面}
    \label{fig:trans_data_save}
    %\vspace{2mm}
  \end{minipage}
\end{figure}

\subsection {通知を受け取る}
通知をタップする.
保存した旅程データがバッチ処理によって災害情報と紐付けられると,push通知が届く.
そしてホーム画面の災害注意予報一覧から注意報が出ている旅程パッケージをタップする.
なお,本アプリにおいては遠隔からの通知を受け取る仕様ではなく,ユーザがアプリを通じて通知を送信する.
本来であれば遠隔から通知を受け取るべきだが,後述する評価実験に関係のない機能であることから本実験では実装を見送った.

\begin{figure}[H]
  \begin{minipage}[b]{0.45\linewidth}
    \centering
    \includegraphics[height=10cm]{./fig/notion.png}
    %\vspace{-3mm}
    \caption{通知の表示}
    \label{fig:notion}
    %\vspace{2mm}
  \end{minipage}
  \begin{minipage}[b]{0.45\linewidth}
    \centering
    \includegraphics[height=10cm]{./fig/unormal_home_screen.png}
    %\vspace{-3mm}
    \caption{災害注意予報通知時のホーム画面}
    \label{fig:unormal_home_screen}
    %\vspace{2mm}
  \end{minipage}
\end{figure}

\subsection {災害注意予報を確認する}
選択した旅行パッケージの災害注意予報が紐付けられている旅程データの一覧を閲覧する.
一覧から旅程データを選択する.
\begin{figure}[H]
  \centering
  \includegraphics[height=10cm]{./fig/advisory_data_list.png}
  %\vspace{-3mm}
  \caption{災害注意予報が出ている旅程データの一覧}
  \label{fig:advisory_data_list}
  %\vspace{2mm}
\end{figure}

\subsubsection {場所データの災害注意予報}
場所データに対する災害注意予報を閲覧する.
提供される災害の種類は雨と風(風雪)である.災害の種類ごとに災害が起こる確率を提供している.
それぞれの災害のリストをタップすると,ストック情報の画面に遷移する.

\begin{figure}[H]
  \centering
  \includegraphics[height=10cm]{./fig/spot_advisory.png}
  %\vspace{-3mm}
  \caption{場所データの災害注意予報画面}
  \label{fig:spot_advisory}
  %\vspace{2mm}
\end{figure}

\subsubsection {交通データの災害注意予報}
交通データに対する災害注意予報を閲覧する.
交通データにおいては,場所データと同じように駅データごとに災害注意予報が提供される.
各駅に対して災害の発生する可能性を運休する可能性として情報を提供している.
さらに,ページ下部には災害時に起こる交通機関の運休の現象や計画運休の現象についての説明がある.

\begin{figure}[H]
  \begin{minipage}[b]{0.45\linewidth}
    \centering
    \includegraphics[height=10cm]{./fig/trans_advisory_1.png}
    %\vspace{-3mm}
    \caption{交通データの災害注意予報画面1}
    \label{fig:trans_advisory_1}
    %\vspace{2mm}
  \end{minipage}
  \begin{minipage}[b]{0.45\linewidth}
    \centering
    \includegraphics[height=10cm]{./fig/trans_advisory_2.png}
    %\vspace{-3mm}
    \caption{交通データの災害注意予報画面2}
    \label{fig:trans_advisory_2}
    %\vspace{2mm}
  \end{minipage}
\end{figure}

\subsection {ストック情報を確認する}
災害のストック情報を閲覧する.
ストック情報は雨と風(風雪)の2種類の災害の情報である.

\subsubsection {雨の災害の情報}
日本における雨に関する災害についての説明が記載されている.
雨に関する災害とは大雨そのものの現象以外に洪水と土砂災害のことである.
各災害についての説明とそれに対する対策喚起の情報が載っている.

\begin{figure}[H]
  \begin{minipage}[b]{0.45\linewidth}
    \centering
    \includegraphics[height=10cm]{./fig/rain_stock_1.png}
    %\vspace{-3mm}
    \caption{雨のストック情報提供画面1}
    \label{fig:rain_stock_1}
    %\vspace{2mm}
  \end{minipage}
  \begin{minipage}[b]{0.45\linewidth}
    \centering
    \includegraphics[height=10cm]{./fig/rain_stock_2.png}
    %\vspace{-3mm}
    \caption{雨のストック情報提供画面2}
    \label{fig:rain_stock_2}
    %\vspace{2mm}
  \end{minipage}
\end{figure}

\subsubsection {風の災害の情報}
日本における風に関する災害についての説明が記載されている.
具体的には台風と高潮,暴風についてである.
各災害についての説明とそれに対する対策喚起の情報が載っている.

\begin{figure}[H]
  \begin{minipage}[b]{0.45\linewidth}
    \centering
    \includegraphics[height=10cm]{./fig/wind_stock_1.png}
    %\vspace{-3mm}
    \caption{風のストック情報提供画面1}
    \label{fig:rain_stock_1}
    %\vspace{2mm}
  \end{minipage}
  \begin{minipage}[b]{0.45\linewidth}
    \centering
    \includegraphics[height=10cm]{./fig/wind_stock_2.png}
    %\vspace{-3mm}
    \caption{風のストック情報提供画面2}
    \label{fig:rain_stock_2}
    %\vspace{2mm}
  \end{minipage}
\end{figure}

% \begin{figure}[H]
%   \centering
%   \includegraphics[height=8cm]{figure32.jpg}
%   %\vspace{-3mm}
%   \caption{場所データの注意情報提供画面の図}
%   \label{fig:activity}
%   %\vspace{2mm}
% \end{figure}

% \subsection {公共交通データの注意情報提供画面の図}

% \begin{figure}[H]
%   \centering
%   \includegraphics[height=8cm]{figure32.jpg}
%   %\vspace{-3mm}
%   \caption{公共交通データの注意情報提供画面の図}
%   \label{fig:activity}
%   %\vspace{2mm}
% \end{figure}

% \subsection {大雨に関するストック情報の提供画面の図}

% \begin{figure}[H]
%   \centering
%   \includegraphics[height=8cm]{figure32.jpg}
%   %\vspace{-3mm}
%   \caption{大雨に関するストック情報の提供画面の図}
%   \label{fig:activity}
%   %\vspace{2mm}
% \end{figure}

% \subsection {風に関するストック情報の提供画面}

% \begin{figure}[H]
%   \centering
%   \includegraphics[height=8cm]{figure32.jpg}
%   %\vspace{-3mm}
%   \caption{風に関するストック情報の提供画面の図}
%   \label{fig:activity}
%   %\vspace{2mm}
% \end{figure}
%
\end{document} % 2章
\documentclass[a4paper,11pt,oneside,openany]{jsbook}
\usepackage{myjlabthesisstyle}
\daigaku{青山学院大学}
\gakubu{社会情報}
\gakka{社会情報学科}
\syubetsu{卒業論文}
\labname{宮治研究室}
\chiefexaminer{宮治~~裕~~教授}

%%%%%%%%%%%%%%%%%%%%%%%%%%%%%%%%%%%%%%%
% ここから先「ここまで個人設定」の範囲に
% 各自の固有の情報を記入して下さい
%%%%%%%%%%%%%%%%%%%%%%%%%%%%%%%%%%%%%%%
\nendo{2023年度}
\teisyutsu{2024年~~1月}
\snum{15387019}
\jname{宮治~~裕}
\thesistitle{宮治研における論文作成について} %タイトルを記入
%\thesissubtitle{\LaTeX の利用} %サブタイトルを記入 ない場合はコメントアウト
%\SUBTtrue %サブタイトル有りの場合 ない場合は,コメントアウト
\SUBTfalse %サブタイトル無しの場合 有る場合は,コメントアウト
%%%%%%%%%% ここまで個人設定 %%%%%%%%%%%%%%

\begin{document}

\chapter{\LaTeX の利用例}
本章では,\LaTeX の利用例について示す.
特に,基本的な記述方法,図の組み込み方,表の組み込み型について,具体例と共に解説する.

\input{sec31basic}
\input{sec32figure}
\input{sec33table}
\input{sec34bib}

%
\end{document} % 3章
\documentclass[a4paper,11pt,oneside,openany]{jsbook}
\usepackage{myjlabthesisstyle}
\daigaku{青山学院大学}
\gakubu{社会情報}
\gakka{社会情報学科}
\syubetsu{卒業論文}
\labname{宮治研究室}
\chiefexaminer{宮治~~裕~~教授}

%%%%%%%%%%%%%%%%%%%%%%%%%%%%%%%%%%%%%%%
% ここから先「ここまで個人設定」の範囲に
% 各自の固有の情報を記入して下さい
%%%%%%%%%%%%%%%%%%%%%%%%%%%%%%%%%%%%%%%
\nendo{2023年度}
\teisyutsu{2024年~~1月}
\snum{15387019}
\jname{宮治~~裕}
\thesistitle{宮治研における論文作成について} %タイトルを記入
%\thesissubtitle{\LaTeX の利用} %サブタイトルを記入 ない場合はコメントアウト
%\SUBTtrue %サブタイトル有りの場合 ない場合は,コメントアウト
\SUBTfalse %サブタイトル無しの場合 有る場合は,コメントアウト
%%%%%%%%%% ここまで個人設定 %%%%%%%%%%%%%%

\begin{document}

\chapter{おわりに}
本章では本研究のまとめと今後の展望について述べる.

\section{まとめ}
本研究は気象庁が発表する早期注意情報を利用し,訪日観光客が風水害の影響を受ける前に情報を提供することで,防災行動を促す効果があることを明らかにする研究であった.
この研究を行なった背景は,訪日観光客が台風などの風水害の現象を想起することが難しく,現状の情報提供方法は日本住民向けであるため,訪日観光客に理解し難いことであった.
原因はストック情報が既知でなく,フロー情報を理解できないことであった.
ストック情報とはこれまでの教育や訓練などで人に蓄積された災害に関する情報である.
フロー情報はある特定の災害に対する災害についての説明と避難の呼びかけの情報である.
その問題に対して,訪日観光客の旅行計画に基づいて訪日中に災害の影響を受けるか監視し,影響がある場合ストック情報とフロー情報を提供するシステムを提案した.
新規性は情報提供するタイミングが余裕を持って行うことと,ストック情報を知らなくても理解しやすい情報を提供していることである.
気象庁が提供している天気予報の情報とシステムが提供する情報を比較することで,比較実験をおこなった.
実験参加者は全て日本人である.
防護動機理論という理論に基づいて,指標は脅威評価の災害の深刻さに対する認知と災害発生の確率に対する認知とした.
過去の災害の情報から,実験用の旅程を2つ作成した.1日先に災害の可能性が認められる旅程と2日先に災害の可能性が認められる旅程である.
実験の結果,どちらの旅程においても天気予報から情報を得た実験参加者より,システムをから情報を得た実験参加者の脅威評価の方が高いと判明した.
さらに,システムはより未来の旅行日程に対して天気予報よりも脅威評価を高まらせる効果がある.
したがって,システムは天気予報と比較するとより高い脅威評価を与える効果があり,事前に防災行動を促すことが見込める.

\section{今後の展望}
今回の実験は日本人を対象としておこなったので,訪日観光客に防災行動を促すという目的を達成できたとは言えない.
今後の展望として,外国人を対象とした評価実験を行う必要がある.
また,システムで提供する情報も外国語化する必要がある.
そして,提供する情報についても文献から確認できた情報のみに基づいて作成されたので,訪日観光客に使用してもらった感想を聞き,精錬していきたい.
%
\end{document} % 4章
\documentclass[a4paper,11pt,oneside,openany]{jsbook}
\usepackage{myjlabthesisstyle}
\daigaku{青山学院大学}
\gakubu{社会情報}
\gakka{社会情報学科}
\syubetsu{卒業論文}
\labname{宮治研究室}
\chiefexaminer{宮治~~裕~~教授}

%%%%%%%%%%%%%%%%%%%%%%%%%%%%%%%%%%%%%%%
% ここから先「ここまで個人設定」の範囲に
% 各自の固有の情報を記入して下さい
%%%%%%%%%%%%%%%%%%%%%%%%%%%%%%%%%%%%%%%
\nendo{2023年度}
\teisyutsu{2024年~~1月}
\snum{15387019}
\jname{宮治~~裕}
\thesistitle{宮治研における論文作成について} %タイトルを記入
%\thesissubtitle{\LaTeX の利用} %サブタイトルを記入 ない場合はコメントアウト
%\SUBTtrue %サブタイトル有りの場合 ない場合は,コメントアウト
\SUBTfalse %サブタイトル無しの場合 有る場合は,コメントアウト
%%%%%%%%%% ここまで個人設定 %%%%%%%%%%%%%%

\begin{document}

\chapter{システム構成図の例}
システム構成図が論理的に描けると,論文そのものの説明もしやすくなる.
ここでは,シスム構成図の例をいくつか記載する.
\begin{figure}[h]
  \centering
  \includegraphics[width=12cm]{VKall.pdf}
  \vspace{-1mm}
  \caption{擬似感性の構成}
  \label{fig:vkall}
  \vspace{5mm}
\end{figure}

\begin{figure}[h]
  \centering
  \includegraphics[width=14cm]{VoiceKANSEIDetector.pdf}
  \vspace{-1mm}
  \caption{音声からの感性同定部}
  \label{fig:VoiceKANSEIDetector}
  \vspace{5mm}
\end{figure}
% \begin{figure}[bt]
%     \centering
%     \includegraphics[width=14cm]{mp2.pdf}
%     \vspace{-1mm}
%     \caption{MMSの内部構成}
%     \label{fig:mp2}
%     \vspace{5mm}
% \end{figure}
%
\end{document} % 5章
%\include{chap6} % 6章
%\include{chap7} % 7章

%%% 付録 -- 必要なければ以下を2行コメントアウト
\appendix
\include{appendixA}
%\include{appendixB} %必要に応じて付録の数を増やす

%\clearpage
%%%%%%%%%% ここまで論文本文 %%%%%%%%%%%%%%


% ************** ここから先の範囲は編集不要 ****************
%%% 参考文献
\bibliographystyle{junsrt}
\bibliography{myrefs}
% myrefs.bib の中はサンプルファイルを参考に記述

\newpage
\printindex
%
\end{document}
