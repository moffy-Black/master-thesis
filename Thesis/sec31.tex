\section{実験目的}
本実験は,3章で説明したシステムが訪日観光客に対して防災行動を促す効果を持つか検証することを目的とする.
防護動機理論に基づいてシステムがユーザの脅威評価に与える効果があるか,気象庁の天気予報が与える効果と比較することで確認する.

\subsection{防護動機理論}
Rogersによって提唱された防護動機理論\cite{1360292618739561728,1570572700020740224}は健康リスクへの個人の対処行動を説明する理論である.
その応用として災害リスクへの対処行動を分析した事例がある.
柿本ら\cite{Kakimoto2014}はこの理論を拡張して予防的避難に対する意識行動をモデル化した.
予防的避難とは夜間に災害発生の恐れがある場合に,日が出ているうちに早めの避難をすることである.
これは日本住民を対象として行われている取り組みであるが,事前に避難勧告するという点では本研究と共通する.\par
本研究ではこのモデルに沿って評価指標を決定した.
モデルは脅威評価,対処評価,非防護反応の3つの要素が防護動機を構成しているとしている.
脅威評価は災害の深刻さに対する認知と災害発生の確率に対する認知で構成されている.
対処評価は災害への対処行動の効果の認知,対処行動をとることのコストの認知,自分が対処行動を成功させることができる自信である自己効力で構成される.
非防護反応は災害について考えないようにする思考回避や,危険性を認めない否認,運命諦観,楽観視,絶望や信仰で構成される.

\subsection{評価指標}
評価指標を脅威評価とする.
本システムによる情報提供で影響が与えるのは脅威評価であり,先行研究で脅威評価はもっとも防護動機に影響を与えるとされているためである.
脅威評価が高まれば高まるほど,防護動機も高まる.したがって,防災行動を促すことにつながる.
対処評価に関しては,実際に旅行に行って現地の状況を把握しないと評価できない.つまり実験では評価することが難しい.
非防護反応に関しては,システムの情報提供によって影響を与えられるものではなく,個人の考え方として捉える.
したがって,本研究では脅威評価を指標とする.

\subsection{気象庁の天気予報}
システムが提供する情報の比較対象として,気象庁の天気予報を情報を用いる.
気象庁は一般的に天気予報と呼ばれている情報を府県天気予報 \footnote{https://www.jma.go.jp/jma/kishou/know/kurashi/yoho.html} のことだと説明している \footnote{https://www.jma.go.jp/jma/kishou/know/faq/faq4.html}.
府県天気予報は2日先までの天気予報を表形式でWeb上に発表している.また,その天気の概況をテキスト形式の情報で提供している.\par
システムの比較対象としてこれらの情報を選択した理由は,台風19号の災害情報等における事前対応に関する訪日外国人調査\cite{Typhoon}の回答で,訪日中に台風襲来を知った情報媒体は日本のテレビやラジオと答えた人の割合が55.5\%と一番多かったからである.
つまり,現状の訪日観光客の約5割は気象庁の天気予報を間接的に受け取っている.
そのため,気象庁の天気予報を比較対象の情報とする.
