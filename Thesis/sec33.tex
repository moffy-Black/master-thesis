\section{検定方法}
検定方法について述べる.Brunner-Munzel検定\cite{1361981469031994624}を利用した.
Brunner-Munzel検定はノンパラメトリック検定の1つで,2標本の母集団分布の同一性を検定するための手法である.
帰無仮説は2標本の母集団から1つずつ標本を取り出したときに,どちらが大きい確率も等しくなることである.
つまり,以下の数式3.1において
\begin{math}
  p=0.5
\end{math}
となることである.対立仮説は2標本の母集団から1つずつ標本を取り出したときに,どちらが大きい確率も等しくならないことであり,
\begin{math}
  p\neq0.5
\end{math}
になることである.

\begin{equation}
  p = P(X < Y) + \frac{1}{2}P(X = Y)
\end{equation}

本研究の実験に置き換えると,XとYは天気予報グループとシステムグループから得られた回答データを表している.
また,効果量は
\begin{math}
  P(X < Y) + \frac{1}{2}P(X = Y)
\end{math}
の計算結果である.
Rで利用できるBrunner-Munzel検定のライブラリ\cite{cranBrunnerMunzel}を利用して検定する.
% すなわち,効果量が高ければ高いほどシステムグループは天気予報グループより脅威評価が高かったと言える.
% 具体的に本研究においては,Rで利用できるBrunner-Munzel検定のライブラリを利用して検定する.
% 井口豊によると,この検定の効果量はCohen's dと以下の表のような対応関係があるとされている.

% \begin{table}[h]
%   \caption{Brunner-Munzel 検定の効果量と Cohen's d の関係}
%   \centering
%   \begin{tabular}{|c|c|c|c|}
%   \hline
%   \multicolumn{1}{|c|}{効果量} &  \multicolumn{1}{c|}{小} &  \multicolumn{1}{c|}{中} &  \multicolumn{1}{c|}{大}\\
%   \hline \hline
%   Cohen's d & 0.2 & 0.5 & 0.8 \\ \hline
%   \shortstack{Stochastic\\ superiority} & 0.56 & 0.64 & 0.71\\ \hline
%   \end{tabular}
%   \label{table:resultEx1}
% \end{table}