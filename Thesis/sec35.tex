\section{考察}
本研究の目的は開発したアプリが防災行動を促すのか,天気予報と比較して確認することであった.
帰無仮説をシステムと天気予報のグループの母集団から1つずつ回答を取り出したときに,どちらが大きい確率も等しくなることとしていた.
対立仮説はシステムと天気予報のグループの母集団から1つずつ回答を取り出したときに,どちらが大きい確率も等しくならないこととしていた.
前述した実験結果から,指標ごとに考察を述べる.その後に日程ごとの結果から考察を述べる.

災害の深刻さに対する認知の検定結果におけるp値は,1日先が\quad\text{$1.504 \times 10^{-2}$},2日先が\quad\text{$2.507 \times 10^{-3}$}とどちらの日程においても小さい値であった.
この結果から,2つのグループの評価に差があることが偶然である確率は2\%以下である.すなわち,システムグループにおける評価の値と天気予報グループにおける評価の値には差がある.
さらに効果量は,1日先が \quad\text{$7.5 \times 10^{-1}$},2日先が \quad\text{$8.333333 \times 10^{-1}$}とどちらの日程においても0.5以上の値であった.
つまり,システムグループの回答の方が天気予報グループの回答よりも,認知が大きいとする回答数が多かったことを示す.
これらのことから,どちらの日程においても天気予報グループより,システムグループの方が災害の深刻さに対する認知は大きい傾向が確認できる.

災害発生の確率に対する認知の検定結果におけるp値は,1日先が\quad\text{$1.432 \times 10^{-1}$}で,2日先が\quad\text{$1.183 \times 10^{-1}$}であった.
この結果から,2つのグループの評価に差があることが偶然である確率は15\%以下である.すなわち,システムグループにおける評価の値と天気予報グループにおける評価の値には差がある.
さらに効果量は,1日先が \quad\text{$6.791667 \times 10^{-1}$},2日先が \quad\text{$7.0 \times 10^{-1}$}とどちらの日程においても0.5以上の値であった.
これらのことから,どちらの日程においても天気予報グループより,システムグループの方が災害発生の確率に対する認知は大きい傾向が確認できる.

1日先の旅程と2日先の旅程に対する2種類の認知を比較すると,どちらの認知においても2日先の旅程に対する効果量の方が高い.
つまり,より未来の旅行日程に対してシステムは天気予報よりも脅威評価を高める効果がある.

総評としてシステムが提供する情報の方が気象庁の天気予報よりも脅威評価を高める効果があったと言える.
したがって,システムは天気予報と比較して,より防災行動を促すことが見込める.
