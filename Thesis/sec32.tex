\section{実験方法}
調査はオンラインで評価を収集できるように設計した.
実験参加者は日本人で19〜25歳の男女22名が参加した.\par
参加者は2つのグループがあり,対応していない.それぞれのグループをシステムグループと天気予報のグループと呼称する.
システムグループは10名,天気予報のグループは12名の参加者が参加した.
参加者は2つの日程の旅程に対して評価をする.
1つ目は実験参加者から見て,明日に予定されている旅程である.システムは5日先までの旅程に対して注意予報を提供している.
その中でも明日に予定されている旅程は1日先の旅程であり,2日から5日先の予報と違って防災行動の判断を先延ばしにできない.
この理由から1つ目の評価対象として選択した.
2つ目は明後日に予定されている旅程データである.2日先の旅程データは2日から5日先の予報の代表の旅程として選択した.
理由は脅威評価を計測するうえで,もっとも時間的な距離が近い2日先の旅程データは脅威評価が最大化するためである.
つまり2日先のデータの評価は3から5日先のデータの評価を含意する.
これら2つの旅程においてそれぞれのグループの脅威評価をシミュレーションすることで得る.\par
次に実験のために作成した旅程データについて述べて,アンケート調査用紙の説明をした後,それぞれのグループごとに実験の方法を述べる.

\subsection{実験に使用した旅程データ}
評価実験を2つのグループにおいておこなううえで,同一の旅程データをもとに評価する必要がある.
実験に使用する旅程データは令和2年台風第10号の被害に遭うように設計した.
令和2年台風第10号のデータが利用した理由は,気象庁の府県天気予報が2020年3月18日11時より新しい形式に変更され,これ以降の風水害で最大の災害であったためである.
そのため,参加者は2020年9月5日にいる仮定の元で,9月6日と9月7日の旅程データについて評価する.
2020年9月6日に鹿児島に行く旅程を,9月7日に福岡に行く旅程を実験の旅程とした.
具体的にこれらの旅程データは2つの条件に基づいて作成した.付近の地形が山,海,川の場所が旅程に含まれていることと鉄道を利用することである.

\subsubsection{鹿児島の旅程}
鹿児島の旅程では海と山の近くの場所として桜島を選定した.
川の近くの場所として川内駅を選んだ.川内駅は川内川の近くにある.
また移動手段にJR鹿児島本線を選んだ.鹿児島中央駅から川内駅の13駅を登録した.

\begin{table}[h]
  \caption{鹿児島に行く旅程}
  \centering
  \begin{tabular}{|c|r|r|r|r|r|}
  \hline
  \multicolumn{1}{|c|}{日付} & \multicolumn{1}{c|}{種別} & \multicolumn{1}{c|}{出発} & \multicolumn{1}{c|}{到着} & \multicolumn{1}{c|}{イベント}\\
  \hline \hline
  9/6(水) & 👣移動 &  & 6:30 & 鹿児島空港\\ \hline
  & 👣移動 & 6:45 & 7:30 & [バス] 鹿児島空港 → 桜島フェリーターミナル\\ \hline
  & 👣移動 & 7:30 & 8:00 & [フェリー] 桜島フェリーターミナル → 桜島港\\ \hline
  & 📷観光 & 8:00 & 14:00 & 桜島\\ \hline
  & 👣移動 & 14:00 & 14:30 & [フェリー] 桜島港 → 桜島フェリーターミナル \\ \hline
  & 👣移動 & 15:00 & 16:00 & [JR鹿児島本線] 鹿児島中央 → 川内\\ \hline
  & 🏨宿泊 & 17:00 &  & クマンジョベース\\ \hline
  \end{tabular}
  \label{table:resultEx1}
\end{table}

\subsubsection{福岡の旅程}
福岡の旅程では海の近くの場所としてシーサイドももち海浜公園が,山の近くの場所として皿倉山が,川の近くの場所として櫛田神社博多総鎮守を選定した.
川は博多川である.
また,移動手段にJR鹿児島本線と福岡地下鉄空港線を選んだ.
JR鹿児島本線は博多から八幡までの18駅を登録した.
福岡地下鉄空港線は福岡空港から西新までの9駅を登録した.

\begin{table}[h]
  \caption{福岡に行く旅程}
  \centering
  \begin{tabular}{|c|r|r|r|r|}
  \hline
  \multicolumn{1}{|c|}{日付} &  \multicolumn{1}{c|}{種別} &  \multicolumn{1}{c|}{出発} &  \multicolumn{1}{c|}{到着} &  \multicolumn{1}{c|}{イベント}\\
  \hline \hline
  9/7(木) & 👣移動 &  & 6:30 & 福岡空港に到着\\ \hline
   & 👣移動 & 6:30 & 7:30 & [福岡市営空港線] 福岡空港駅 → 西新\\ \hline
   & 📷観光 & 7:30 & 9:30 & シーサイドももち海浜公園\\ \hline
   & 👣移動 & 9:30 & 10:00 & [福岡市地下鉄空港線] 西新 → 中洲川端駅\\ \hline
   & 📷観光 & 10:00 & 12:00 & 櫛田神社\\ \hline
   & 👣移動 & 12:00 & 12:10 & [福岡市地下鉄空港線] 中洲川端駅 → 博多\\ \hline
   & 👣移動 & 12:10 & 13:30 & [JR鹿児島本線] 博多 → 八幡\\ \hline
   & 📷観光 & 14:00 & 16:00 & 皿倉山\\ \hline
   & 🏨宿泊 & 18:00 &  & 千草ホテル\\ \hline
  \end{tabular}
  \label{table:resultEx1}
\end{table}

\newpage

\subsection{アンケート調査用紙}
評価指標である脅威評価を計測するためのアンケート調査用紙について説明する.
調査用紙はGoogleフォームによって作成した.
脅威評価を構成する災害の深刻さに対する認知と災害発生の確率に対する認知を質問により,5段階のリッカート尺度で評価する.
質問文はそれぞれの旅程に対して「旅行中に災害に巻き込まれたとしたら,被害は大きいと思う」と「旅行中に災害の被害に遭うと思う」である.
回答項目は共通であり,「そう思わない」「どちらかというとそう思わない」「どちらでもない」「どちらかというとそう思う」「そう思う」である.
これらは順序尺度であり,順番に1から5までの離散値を評価の値として割り当てた.

\subsection{システムグループ}
システムグループの実験手順は以下の通りである.

\begin{quote}
  \begin{enumerate}
    \item 実験承諾する
    \item 実験の説明を受ける
    \item 鹿児島の旅程に対する災害注意予報を閲覧する
    \item 鹿児島の旅程に対するアンケート項目を回答する
    \item 福岡の旅程に対する災害注意予報を閲覧する
    \item 福岡の旅程に対するアンケート項目を回答する
  \end{enumerate}
\end{quote}

\subsubsection{実験承諾をする}
実験参加前,実験参加者は実験承諾書に同意した.
実験の概要と個人情報の取り扱いについて確認した後に,承諾の旨をメールで意思表示した.

\subsubsection{実験の説明を受ける}
まず,実験参加者は実験の目的と方法について説明を受ける.
その次に実験に使用する鹿児島の旅程データと,福岡の旅程データについて確認する.
旅程データはGoogleスプレッドシートで作成した表形式のデータとGoogleマイマップで作成した地図形式のデータである.
表形式のデータは観光と移動の2種類のイベントで構成した.
観光のイベントは観光の場所の名称と,予定時刻,観光の内容,観光地のURLを記載した.
移動のイベントは移動の出発地点と到達地点,予定時刻,移動手段について記載した.
地図形式のデータは移動経路を線で,観光地をピンで地図上に表現した.

\subsubsection{鹿児島の旅程に対する災害注意予報を閲覧する}
実験参加者はシステムを使用して,災害注意予報の情報を確認する.
災害注意予報は2020年9月5日に実際に気象庁から発表された1日先までの早期注意情報をもとに表示した.
鹿児島の旅程の災害注意予報と雨,風の災害についてのストック情報を閲覧する.

\subsubsection{鹿児島の旅程に対するアンケート項目を回答する}
実験参加者はGoogleフォームで作成したアンケートに回答する.
鹿児島の旅程に対して,「旅行中に災害に巻き込まれたとしたら,被害は大きいと思う」と「旅行中に災害の被害に遭うと思う」の質問に回答する.
回答項目は,「そう思わない」「どちらかというとそう思わない」「どちらでもない」「どちらかというとそう思う」「そう思う」である.

\subsubsection{福岡の旅程に対する災害注意予報を閲覧する}
実験参加者はシステムを使用して,鹿児島の旅程と同様に福岡の旅程の災害注意予報の情報を確認する.
災害注意予報は2020年9月5に実際に気象庁から発表された2日先以降の早期注意情報をもとに表示した.
そして,雨,風の災害についてのストック情報を閲覧する.ストック情報は鹿児島の旅程で確認したものと同様である.

\subsubsection{福岡の旅程に対するアンケート項目を回答する}
実験参加者はGoogleフォームで作成したアンケートに回答する.
福岡の旅程に対して,「旅行中に災害に巻き込まれたとしたら,被害は大きいと思う」と「旅行中に災害の被害に遭うと思う」の質問に回答する.
回答項目は,「そう思わない」「どちらかというとそう思わない」「どちらでもない」「どちらかというとそう思う」「そう思う」である.

\subsection{天気予報グループ}
天気予報グループの実験手順は以下の通りである.

\begin{quote}
  \begin{enumerate}
    \item 実験承諾する
    \item 実験の説明を受ける
    \item 鹿児島の旅程に対する天気予報を閲覧する
    \item 鹿児島の旅程に対するアンケート項目を回答する
    \item 福岡の旅程に対する天気予報を閲覧する
    \item 福岡の旅程に対するアンケート項目を回答する
  \end{enumerate}
\end{quote}

\subsubsection {実験承諾する}
実験参加者は実験参加前に実験承諾書に同意した.
実験の概要と個人情報の取り扱いについて確認した後に,承諾の旨をメールで意思表示した.

\subsubsection {実験の説明を受ける}
システムグループと同様の説明を受ける.

\subsubsection {鹿児島の旅程に対する天気予報を閲覧する}
実験参加者は天気予報から情報を確認する.
天気予報は2020年9月5日に実際に気象庁から発表された府県天気予報と天気概況をもとに表示する.
参加者は9月6日に対する,天気や風,波の予報と降水確率,気温について表形式の情報を得る.
そして,天気の概要を閲覧する.

\subsubsection {鹿児島の旅程に対するアンケート項目を回答する}
実験参加者はGoogleフォームで作成したアンケートに回答する.
鹿児島の旅程に対して,「旅行中に災害に巻き込まれたとしたら,被害は大きいと思う」と「旅行中に災害の被害に遭うと思う」の質問に回答する.
回答項目は,「そう思わない」「どちらかというとそう思わない」「どちらでもない」「どちらかというとそう思う」「そう思う」である.

\subsubsection {福岡の旅程に対する天気予報を閲覧する}
実験参加者は天気予報から情報を確認する.
情報は2020年9月5日に実際に気象庁から発表された府県天気予報と天気概況をもとに表示されている.
参加者は9月7日に対する,天気や風,波の予報と降水確率,気温について表形式の情報を得る.
そして,天気の概要を閲覧する.

\subsubsection {福岡の旅程に対するアンケート項目を回答する}
実験参加者はGoogleフォームで作成したアンケートに回答する.
福岡の旅程に対して,「旅行中に災害に巻き込まれたとしたら,被害は大きいと思う」と「旅行中に災害の被害に遭うと思う」の質問に回答する.
回答項目は,「そう思わない」「どちらかというとそう思わない」「どちらでもない」「どちらかというとそう思う」「そう思う」である.