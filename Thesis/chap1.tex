\documentclass[a4paper,11pt,oneside,openany]{jsbook}
\usepackage{myjlabthesisstyle}
\daigaku{青山学院大学}
\gakubu{社会情報}
\gakka{社会情報学科}
\syubetsu{卒業論文}
\labname{宮治研究室}
\chiefexaminer{宮治~~裕~~教授}

%%%%%%%%%%%%%%%%%%%%%%%%%%%%%%%%%%%%%%%
% ここから先「ここまで個人設定」の範囲に
% 各自の固有の情報を記入して下さい
%%%%%%%%%%%%%%%%%%%%%%%%%%%%%%%%%%%%%%%
\nendo{2023年度}
\teisyutsu{2024年~~1月}
\snum{15387019}
\jname{宮治~~裕}
\thesistitle{宮治研における論文作成について} %タイトルを記入
%\thesissubtitle{\LaTeX の利用} %サブタイトルを記入 ない場合はコメントアウト
%\SUBTtrue %サブタイトル有りの場合 ない場合は,コメントアウト
\SUBTfalse %サブタイトル無しの場合 有る場合は,コメントアウト
%%%%%%%%%% ここまで個人設定 %%%%%%%%%%%%%%

\begin{document}

\chapter{はじめに}
本論文では,○○○を△△△することにより,□□を明らかとする研究について記述する.

まず,本研究をおこなう背景となった事柄について述べる.
次に,研究目的の詳細を記述した後,類似研究との相違や関連研究とのつながりについて解説する.
また,次章以降の本論文の構成についてその概略を述べる.

\section{背景}
「背景」には,研究の目的の妥当性を示す事項を説明する.
個人的な興味や関心を書くのではなく,客観的な視点で記述する.
つまり,その説明には参考文献やデータを参照することが必要となる.

なお,背景をあまり詳しく書きすぎると,2章や3章などで書く内容が無くなったり重複するおそれがある.
研究の目的の妥当性につながる程度の内容(詳細さ)でかまわない.

\section{研究目的}
背景によって,研究の大きな目的が導かれる.
その大きな目的を正確に定義した後,本研究にて実際にターゲットとする目的を詳細に記述する\footnote{大きな目的は1年間の研究ではカバーしきれないため}.

また,背景にて実際の詳細なターゲットの必要性を示した場合には,それの詳細な条件を記載する.

\section{関連研究}
類似研究(同じような研究)とは,どこが違うのか(ターゲット,手法,想定結果など)を述べる必要がある.
また,参考にする先行研究(他組織の研究でも良い)とどのような関連性があるのかを述べる.

場合によっては,関連研究が研究目的より先に書いてあった方が「ながれ」が良い場合もある.
また,関連研究を背景の中に入れてしまった方が良いケースもある.
これらについては,文章を書きながら,判断するしかない.

\section{論文構成}
論文構成では,2章以降の大まかな記述内容の流れを示す.

たとえば,以下の様に記述する.
2章では,本研究にて活用した技術や関連サービスについて解説する.
3章では,提案・構築したシステムについて詳説する.
4章では,システムの有用性を検証する為に行った実験について記述する.
最後に5章において,本研究についてまとめ,今後の課題について述べる.


%%%%%%%%%%%%フォーマットの確認開始%%%%%%%%%%%%
\newpage
%\clearpage
\noindent
一二三四五六七八九零一二三四五六七八九零一二三四五六七八九零一二三四五\\
二\\
三\\
四\\
五\\
六\\
七\\
八\\
九\\
零\\
一\\
二\\
三\\
四\\
五\\
六\\
七\\
八\\
九\\
零  行数と列数の設定テスト 30行×35文字 = 1050文字/ページ\\
一\\
二\\
三\\
四\\
五\\
六\\
七\\
八\\
九\\
零
%%%%%%%%%%%%フォーマットの確認終了%%%%%%%%%%%%

%
\end{document}
