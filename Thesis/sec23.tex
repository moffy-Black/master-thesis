\section{settings.tex: 論文の設定情報を記述}
settings.tex には,各自の個人情報や論文のタイトルなどを設定する.

\subsection{各自の情報設定}
各自の情報を設定する際には,サブタイトルの有り/無しで設定事項が異なることに注意をする必要がある.
それぞれの方法について以下に記述する.
また,これらの作業が終わった時点で,本配布スタイルパッケージの動作確認をすることをおすすめする.

\subsubsection{サブタイトル有りの場合}
配布したファイルは,サブタイトルがある場合のサンプルになっている.
各自の 年度,提出年月,学籍番号,氏名,タイトル,サブタイトルを所定の命令内に記入する.
\begin{breakbox}
{\small
%footnotesize
\begin{verbatim}
\nendo{2013年度}
\teisyutsu{2014年~~1月}
\snum{15387019}
\jname{宮治 裕}
\thesistitle{宮治研における論文作成について} %タイトルを記入
\thesissubtitle{\LaTeX の利用} %サブタイトルを記入 ない場合はコメントアウト
\SUBTtrue %サブタイトル有りの場合 ない場合は,コメントアウト
%\SUBTfalse %サブタイトル無しの場合 有る場合は,コメントアウト
\end{verbatim}
}
\end{breakbox}

\subsubsection{サブタイトル無しの場合}
サブタイトル有りの場合と比較して2箇所の変更が必要である.
サブタイトルを記入する命令の先頭部分に \% 記号を入れ,コメントアウト状態にする.

\begin{breakbox}
{\small
\begin{verbatim}
%\thesissubtitle{\LaTeX の利用} %サブタイトルを記入 無い場合は,コメントアウト
\end{verbatim}
}
\end{breakbox}
もう一つは,その直下の2行
\begin{breakbox}
{\small
\begin{verbatim}
\SUBTtrue %サブタイトル有りの場合 無い場合は,コメントアウト
%\SUBTfalse %サブタイトル無しの場合 有る場合は,コメントアウト
\end{verbatim}
}
\end{breakbox}
以下の様に変更する.
\begin{breakbox}
{\small
\begin{verbatim}
%\SUBTtrue %サブタイトル有りの場合 無い場合は,コメントアウト
\SUBTfalse %サブタイトル無しの場合 有る場合は,コメントアウト
\end{verbatim}
}
\end{breakbox}

以上の設定で,表紙と各ページのヘッダ・フッタの情報が自動的に設定され,書式が整えられる.
\begin{boxnote}
\LaTeX では 「\verb+%+」はコメントを意味し,この記号から改行コードまでをコメントアウト状態として処理する.
\end{boxnote}
であることに注意すること.

\subsection{スタイルパッケージの動作確認}
サブタイトルの有り/無しに応じて適切に設定ができた段階で,一度各自の環境下でスタイルパッケージが正常動作することを確認して欲しい.
正常動作した場合には,本ファイルとほぼ同様の中身で,表紙と各ページのヘッダとフッタが各自の設定した情報が記載されたPDFファイルが出来上がるはずである.

まず,Macintoshの場合について記す.
各自のホームディレクトリ中のDropboxフォルダ内に,本スタイルパッケージが展開されている場合を前提として記述する.
\begin{enumerate}
\item まず,ターミナルを開く
\item 以下のコマンドを入力し,スタイルパッケージのあるフォルダに移動
\footnote{ここで \verb+$+記号は,コマンドプロンプトを表すため,入力しないように.}
\begin{screen}
{\small
\begin{verbatim}
 $ cd ~/Dropbox/Thesis
\end{verbatim}
}
\end{screen}

\item そこで,バッチファイル \verb+mklatex.bat+ を実行
\begin{screen}
{\small
\begin{verbatim}
 $ ./mklatex.bat
\end{verbatim}
}
\end{screen}

\item main.pdfファイルが作成され,プレビュー画面が自動で表示される
\item[\textbf{注}] mklatex.bat が実行できないというようなエラーが出た場合には,最初の一回だけ(次回から不要)以下の命令を入力する
\begin{screen}
{\small
\begin{verbatim}
 $ chmod 755 ./mklatex.bat
\end{verbatim}
}
\end{screen}
\end{enumerate}

正常動作しなかった場合には,出来上がった main.log ファイルを宮治に送付して欲しい.

Windowsの場合には,コマンドプロンプトを開き,目的のフォルダに移動し,バッチファイル(winmklatex.bat)を起動する.
\begin{screen}
{\small
\begin{verbatim}
 $ cd c:\My Documents\Dropbox\Thesis
 $ winmklatex.bat
\end{verbatim}
}
\end{screen}
main.pdfファイルができるので,エクスプローラからファイルをダブルクリックしてAcrobat Reader にて確認して欲しい.
