% 要旨には,論文の要約を記述します.要約と言っても全ての章や項目を均等に縮めるのではなく,必要な項目に絞って端的に示します.

% 論文の概要が,要旨に書かれた文章のみで伝わるようにしなければなりません.
% したがって,少なくとも「ざっくりとした背景」「研究の目的」「他の研究との違い/関わり」「構築したシステム/提案した事項」「結果/得られた結論」が書かれている必要があります.

近年,コロナウィルスの流行により日本における訪日観光客数は減少していたが,回復しつつある.
日本は災害が多く発生する国であり,訪日観光客に対する災害対策の研究には需要がある.
本研究は訪日観光客向けの災害対策の1つとして,災害の情報を提供する情報システムを提案する.
研究の目的は予測できる災害に対して,訪日観光客が災害の影響を受ける前に防災行動を促すことを目的とする.
予測できる災害とは気象に基づく災害であり,本研究では風水害としている.
現状,訪日観光客は災害の情報の多くを日本のテレビやラジオから受け取っている.
これは日本住民に向けた情報であり,一部の訪日観光客は災害の情報を正しく理解できていない.
具体的にこの情報の問題点は,情報を受け取る人がある程度日本の災害に対して知識を持っている前提で提供されていることである.
そのため,事前に予測できる災害に対して認知をしていたのにも関わらず,災害の現象を想起できない.
被害にあってから防災行動を取ることを余儀なくされてしまう現象が確認されている.
本研究では訪日観光客が予想できる災害の現象を想像するのに必要とする知識を定義する.
さらに,訪日観光客の旅程をもとに災害の動向を監視することで,災害の危険性をある程度認められる場合に適切な災害の情報を提供する.
本研究の新規性は訪日観光客が必要とする情報を定義した点,旅程に基づいて災害情報を提供する点である.
提案するシステムは訪日観光客が旅程を入力するアプリである.
気象庁の早期注意情報をもとに,旅程が災害の被害にあう可能性が予測されたとき,災害の情報を訪日観光客に通知する.
このシステムが提供する情報と気象庁の天気予報の情報を比較する実験をおこなった.
評価指標は防護動機理論に基づいて決定され,脅威評価とした.
比較実験の結果,システムから得られた脅威評価の方が天気予報から得られた評価よりも高いことが判明した.