\documentclass[a4paper,11pt,oneside,openany]{jsbook}
\usepackage{myjlabthesisstyle}
\daigaku{青山学院大学}
\gakubu{社会情報}
\gakka{社会情報学科}
\syubetsu{卒業論文}
\labname{宮治研究室}
\chiefexaminer{宮治~~裕~~教授}

%%%%%%%%%%%%%%%%%%%%%%%%%%%%%%%%%%%%%%%
% ここから先「ここまで個人設定」の範囲に
% 各自の固有の情報を記入して下さい
%%%%%%%%%%%%%%%%%%%%%%%%%%%%%%%%%%%%%%%
\nendo{2024年度}
\teisyutsu{2024年~~1月}
\snum{38122001}
\jname{黒川~~皇輝}
\thesistitle{訪日観光客を対象とした風水害注意情報提供システム} %タイトルを記入
%\thesissubtitle{\LaTeX の利用} %サブタイトルを記入 ない場合はコメントアウト
%\SUBTtrue %サブタイトル有りの場合 ない場合は,コメントアウト
\SUBTfalse %サブタイトル無しの場合 有る場合は,コメントアウト
%%%%%%%%%% ここまで個人設定 %%%%%%%%%%%%%%

\begin{document}

\chapter{おわりに}
本章では本研究のまとめと今後の展望について述べる.

\section{まとめ}
本研究は気象庁が発表する早期注意情報を利用し,訪日観光客が風水害の影響を受ける前に情報を提供することで,防災行動を促す効果があることを明らかにする研究であった.
この研究をおこなった背景は,訪日観光客が台風などの風水害の現象を想起することが難しく,現状の情報提供方法は日本住民向けであるため,訪日観光客に理解し難いことであった.
原因はストック情報が既知でなく,フロー情報を理解できないことであった.
ストック情報とはこれまでの教育や訓練などで人に蓄積された災害に関する情報である.
フロー情報はある特定の災害に対する災害についての説明と避難の呼びかけの情報である.

その問題に対して,訪日観光客の旅行計画に基づいて訪日中に災害の影響を受けるか監視し,影響がある場合ストック情報とフロー情報を提供するシステムを提案した.
新規性は情報提供するタイミングが余裕を持っておこなうことと,ストック情報を知らなくても理解しやすい情報を提供していることである.

気象庁が提供している天気予報の情報とシステムが提供する情報を比較することで,比較実験をおこなった.
実験参加者は全て日本人である.
防護動機理論という理論に基づいて,指標は脅威評価とした.
アンケート調査をおこない,脅威評価を構成する災害の深刻さに対する認知と災害発生の確率に対する認知についての評価を収集した.
過去の災害の情報から,実験用の旅程を2つ作成した.1日先に災害の可能性が認められる旅程と2日先に災害の可能性が認められる旅程である.
実験の結果,どちらの旅程においても天気予報から情報を得た実験参加者より,システムをから情報を得た実験参加者の脅威評価の方が高いと判明した.
さらに,システムはより未来の旅行日程に対して天気予報よりも脅威評価を高まらせる効果がある.
したがって,システムは天気予報と比較するとより高い脅威評価を与える効果があり,事前に防災行動を促すことが見込める.

\section{今後の展望}
今回の実験は日本人を対象としておこなったので,訪日観光客に防災行動を促すという目的を達成できたとは言えない.
今後の展望として,外国人を対象とした評価実験をおこなう必要がある.
また,システムで提供する情報も外国語化する必要がある.
そして,提供する情報についても文献から確認できた情報のみに基づいて作成されたので,訪日観光客に使用してもらった感想を聞き,洗練していきたい.
%
\end{document}